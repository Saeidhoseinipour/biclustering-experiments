
%----------------------------------------------------------------------------------------
%	PACKAGES AND OTHER DOCUMENT CONFIGURATIONS
%----------------------------------------------------------------------------------------

\documentclass[paper=a4, fontsize=11pt]{scrartcl} % A4 paper and 11pt font size

\usepackage[T1]{fontenc} % Use 8-bit encoding that has 256 glyphs
\usepackage{fourier} % Use the Adobe Utopia font for the document - comment this line to return to the LaTeX default
\usepackage[english]{babel} % English language/hyphenation
\usepackage{amsmath,amsfonts,amsthm} % Math packages

\usepackage{sectsty} % Allows customizing section commands
\allsectionsfont{\centering \normalfont\scshape} % Make all sections centered, the default font and small caps

\usepackage{fancyhdr} % Custom headers and footers
\pagestyle{fancyplain} % Makes all pages in the document conform to the custom headers and footers
\fancyhead{} % No page header - if you want one, create it in the same way as the footers below
\fancyfoot[L]{} % Empty left footer
\fancyfoot[C]{} % Empty center footer
\fancyfoot[R]{\thepage} % Page numbering for right footer
\renewcommand{\headrulewidth}{0pt} % Remove header underlines
\renewcommand{\footrulewidth}{0pt} % Remove footer underlines
\setlength{\headheight}{13.6pt} % Customize the height of the header

\numberwithin{equation}{section} % Number equations within sections (i.e. 1.1, 1.2, 2.1, 2.2 instead of 1, 2, 3, 4)
\numberwithin{figure}{section} % Number figures within sections (i.e. 1.1, 1.2, 2.1, 2.2 instead of 1, 2, 3, 4)
\numberwithin{table}{section} % Number tables within sections (i.e. 1.1, 1.2, 2.1, 2.2 instead of 1, 2, 3, 4)

\setlength\parindent{0pt} % Removes all indentation from paragraphs - comment this line for an assignment with lots of text

%----------------------------------------------------------------------------------------
%	TITLE SECTION
%----------------------------------------------------------------------------------------

\newcommand{\horrule}[1]{\rule{\linewidth}{#1}} % Create horizontal rule command with 1 argument of height

% \title{	
% \normalfont \normalsize 
% \textsc{university, school or department name} \\ [25pt] % Your university, school and/or department name(s)
% \horrule{0.5pt} \\[0.4cm] % Thin top horizontal rule
% \huge Assignment Title \\ % The assignment title
% \horrule{2pt} \\[0.5cm] % Thick bottom horizontal rule
% }

% \author{John Smith} % Your name

% \date{\normalsize\today} % Today's date or a custom date

\begin{document}

% \maketitle % Print the title q

\section{Co-clustering by Block Value Decomposition}

This is a co-clustering algorithm called Block Value Decomposition (BVD) based on Nonnegative Matrix Factorization (NMF) technique. The goal is to find a factorization for the data matrix $X \in \mathbb{R}_{+}^{N \times M}$, where $N$ is the number of objects, $M$ is the number of features of these objects and the factorization takes the form $$X \approx USV^T$$, where $U \in \mathbb{R}_{+}^{N \times L}$ is a matrix of rows factors representing features clusters, $S \in \mathbb{R}_{+}^{L \times K}$ is a block matrix representing how blocks are related, and $V \in \mathbb{R}_{+}^{M \times K}$ is a matrix of columns factors representing rows clusters.

This algorithm solves the following optimization problem:
$$\textit{min } ||X - USV^T||^2 \textit{ s.t. } U \geq 0, S \geq 0, V \geq 0$$

The optimization problem can be solved using Lagrange multipliers ($\lambda$), optimizing the following Lagrange function:
$${\cal L} = |X - USV^T||^2 - tr(\lambda_1U^T) - tr(\lambda_2S^T) - tr(\lambda_3V^T)$$

Then ${\cal L}$ must satisfy the K.K.T. conditions:
$$\frac{\partial {\cal L}}{\partial U} = 0$$
$$\frac{\partial {\cal L}}{\partial S} = 0$$
$$\frac{\partial {\cal L}}{\partial V} = 0$$
$$\lambda_1 \odot U = 0$$
$$\lambda_2 \odot S = 0$$
$$\lambda_3 \odot V = 0$$

Solving the derivatives and equal them to $0$, is possible to solve the optimization problem by applying gradient ascending on ${\cal L}$ with the following update rules:
$$U \gets U \odot \frac{XVS^T}{USV^TVS^T}$$
$$V \gets V \odot \frac{U^TXV}{U^TUSV^TV}$$
$$S \gets S \odot \frac{S^TU^TX}{S^TU^TUSV^T}$$


\section{Fast Nonnegative Matrix Tri factorization}

In this case, the goal is to optimize the following problem:
$$\textit{min } ||X - USV^T||^2 \textit{ s.t. } U \in \Psi^{n \times k}, S \in \mathbb{R}_{+}^{l \times k}, V \in \Psi^{m \times l}$$ where $U$ and $V$ turns into cluster indicator matrices, with vectors $\vec{u_i}$ and $\vec{v_j}$ that contains $1s$ in only one position, indicating the cluster that that this vector belongs, and $0s$ in the rest.

Similar to the other algorithm, it optimizes $S$ with a multiplicative update rule and the following subproblems:
$$S \gets (U^TU)^{-1}U^TXV(V^TV)^{-1}$$
$$v_{ij} \left\{
\begin{array}{ll}
        1 & j = \textit{argmin}_l ||\vec{x_i} - \vec{\tilde{u_l}}||^2 \\
        0 & \textit{otherwise}
    \end{array}
\right.$$
$$u_{ij} \left\{
\begin{array}{ll}
        1 & i = \textit{argmin}_k ||\vec{x_j} - \vec{\tilde{v_k}}||^2 \\
        0 & \textit{otherwise}
    \end{array}
\right.$$
where $\tilde{U} = US$ and $\tilde{V} = SV^T$


\section{Overlapping Orthogonal Nonnegative Matrix Tri Factorization}

This is a proposal algorithm that aims to solve the following problem:
$$\textit{min } ||X - UV'||^{2}_{F}$$
$$\textit{s.t. } U^TU = I$$
$$\left[ \begin{array}{c} V^{(1)} \\ V^{(2)} \\ \vdots \\ V^{(k)} \end{array} \right]^T\left[ \begin{array}{c} V^{(1)} \\ V^{(2)} \\ \vdots \\ V^{(k)} \end{array} \right] = I$$
$$\left[ \begin{array}{c} V^{(1)} \\ V^{(2)} \\ \vdots \\ V^{(k)} \end{array} \right] \geq 0$$
$$U, S \geq 0$$
where $V^{(c)} \in \mathbb{R}^{M \times L}$ and $V^{'}_{c \cdot} = S_{c \cdot} V^{(c)^T}$, $\forall c \in \{1, \dots, k\}$.

This way the objects (lines) in $X$ can belong to multiple clusters.

\end{document}
